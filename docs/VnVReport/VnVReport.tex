\documentclass[12pt, titlepage]{article}

\usepackage{booktabs}
\usepackage{tabularx}
\usepackage{hyperref}
\usepackage{longtable}
\hypersetup{
    colorlinks,
    citecolor=black,
    filecolor=black,
    linkcolor=red,
    urlcolor=cyan
}
\usepackage[round]{natbib}

\input{../Comments}

\begin{document}

\title{Test Report: Truss} 
\author{Ting-Yu Wu}
\date{\today}
	
\maketitle

\pagenumbering{roman}

\section{Revision History}

\begin{tabularx}{\textwidth}{p{3.5cm}p{2cm}X}
\toprule {\bf Date} & {\bf Version} & {\bf Notes}\\
\midrule
December 11, 2020 & 1.0 & Initial version of VnV report\\
\bottomrule
\end{tabularx}

~\newpage

\section{Symbols, Abbreviations and Acronyms}

\renewcommand{\arraystretch}{1.2}
\begin{tabular}{l l} 
  \toprule		
  \textbf{symbol} & \textbf{description}\\
  \midrule 
  T & Test\\
  \bottomrule
\end{tabular}\\

For the other symbols, abbreviations and acronyms, see 
\href{https://github.com/tingyuw/cas741/blob/master/docs/SRS/SRS.pdf}{SRS}.

\newpage

\tableofcontents

\listoftables %if appropriate

\newpage

\pagenumbering{arabic}

This document is a report on the results of a testing suite for Truss. Detailed 
descriptions of the tests executed can be found in 
\href{https://github.com/tingyuw/cas741/blob/master/docs/VnVPlan/VnVPlan.pdf}{VnV
 Plan}
\section{Functional Requirements Evaluation} \label{funcreq}
All the functional requirements have been met.
\section{Nonfunctional Requirements Evaluation}

\subsection{Accuracy and Verifiability} \label{accver}
The outputs generated by Truss were compared to the solutions from 
\url{https://skyciv.com/free-truss-calculator/} with the same input data.
\subsection{Understandability} \label{under}
\begin{longtable}{l l}
	\begin{tabular}{l l} 
		\toprule		
		\textbf{Questions} & \textbf{Answer}\\
		\midrule 
		Consistent indentation and formatting style? & yes\\
		Explicit identification of a coding standard? & yes\\
		Are the code identifiers consistent, distinctive, and meaningful? & 
		yes \\
		Are constants (other than 0 and 1) hard-coded into the program? & 
		no \\
		Comments are clear, indicate what is being done, not how? & yes \\
		Is the name/URL of any algorithms used mentioned? & yes \\
		Parameters are in the same order for all functions? & yes\\
		Is code modularized? & yes \\
		Descriptive names of source code files? & yes \\
		Is a design document provided? & no \\
		Overall impression? & 8 \\
		\bottomrule
		\caption{Understandability grade sheet} \label{Undgradesheet} \\
	\end{tabular}\\
\end{longtable}
\subsection{Portability} \label{port}
The system can be performed on different operating systems with a terminal and 
python 3.8.

\subsection{Maintainability} \label{main}
The completeness of the documents of multiple versions and issue tracking are 
well-organized on the Truss GitHub repo  
\url{https://github.com/tingyuw/cas741}.

\subsection{Reliability} \label{relia}
The software didn't break during the execution, and it takes less than one 
second to perform.

\section{Comparison to Existing Implementation}	
The comparison will be done between Truss and Truss Calculator
\url{https://skyciv.com/free-truss-calculator/}.

\section{Unit Testing}
The actual implementation of the unit tests can be found in
\href{https://github.com/tingyuw/cas741/blob/master/docs/VnVPlan/VnVPlan.pdf}{VnV
 Plan} section 6. All test cases are performed by test classes built with the 
 help of Pytest. All tests succeed.

\section{Changes Due to Testing}
No changes have been done. However, a problem was found during the 
implementation. We can't let users decide both distances between joints and 
angles of truss memebers because some values can't properly position a truss 
structure. This problem should be solved in the future.

\section{Automated Testing}
Tools used for automated testing are mentioned in 
\href{https://github.com/tingyuw/cas741/blob/master/docs/VnVPlan/VnVPlan.pdf}{VnV
 Plan} section 4.5. All system and unit tests passed. A Travis CI build for the 
code is in conjunction with Drasil. Here is the link to see the build 
\url{https://travis-ci.com/github/tingyuw/Drasil}. It is passed.

\section{Trace to Requirements}
The purpose of the traceability matrices is to provide easy references on what
has to be additionally modified if a certain component is changed. Table 
\ref{Tbltrace} shows the dependencies between the test cases and the 
requirements. Requirements can be found in 
\href{https://github.com/tingyuw/cas741/blob/master/docs/SRS/SRS.pdf}{SRS}.
\begin{table}[h!]
	\centering
	\begin{tabular}{l l} 
		\toprule		
		\textbf{Requirements} & \textbf{Test section}\\
		\midrule 
		R1 & section \ref{funcreq}\\
		R2 & section \ref{funcreq}\\
		R3 & section \ref{funcreq}\\
		R4 & section \ref{funcreq} \\
		R5 & section \ref{funcreq} \\
		NFR1 & section \ref{accver}\\
		NFR2 & section \ref{accver}\\
		NFR3 & section \ref{under}\\
		NFR4 & section \ref{port}\\
		NFR5 & section \ref{main}\\
		NFR6 & section \ref{relia}\\
		\bottomrule
	\end{tabular}\\
	
	\caption{Traceability Between Test Cases and Requirements} 
	\label{Tbltrace}
\end{table}
		
\section{Trace to Modules}	
The purpose of the traceability matrices is to provide easy references on what
has to be additionally modified if a certain component is changed. Table 
\ref{Tbltracemodule} shows the dependencies between the test cases and the 
modules.
\begin{table}[h!]
	\centering
	\begin{tabular}{l l} 
		\toprule		
		\textbf{Requirements} & \textbf{Test section}\\
		\midrule 
		Input Parameters Module & \ref{funcreq}, \ref{accver}, \ref{under}, 
		\ref{main}\\
		Input Constraints Module & \ref{funcreq}, \ref{accver}, \ref{under}, 
		\ref{main}\\
		Calculation Module & \ref{funcreq}, \ref{accver}, \ref{under}, 
		\ref{main}\\
		Output Module & \ref{funcreq}, \ref{accver}, \ref{under}, \ref{main} \\
		\bottomrule
	\end{tabular}\\
	
	\caption{Traceability Between Test Cases and modules} 
	\label{Tbltracemodule}
\end{table}	

\section{Code Coverage Metrics}
The following code coverage of each module is measured by Coverage.py.
\begin{table}[h!]
	\centering
	\begin{tabular}{l l} 
		\toprule		
		\textbf{Modules} & \textbf{Code coverage}\\
		\midrule 
		Calculations.py & 100\% \\
		Constants.py & 100\% \\
		Control.py & 100\% \\
		InputConstraints.py & 15\% \\
		InputParameters.py & 100\% \\
		OutputFormat.py & 76\% \\
		\bottomrule
	\end{tabular}\\
	
	\caption{Code coverage of each module} 
	\label{Tblcodecov}
\end{table}	

Note that the purpose of the InputConstraints.py is to verify the inputs and 
display an error message if any component is out of bounds. Therefore, the code 
coverage of this module is low for valid input sets.

\newpage

\nocite{*}
\bibliographystyle {plainnat}
\bibliography {ref}

\end{document}