\documentclass[12pt, titlepage]{article}

\usepackage{booktabs}
\usepackage{tabularx}
\usepackage{hyperref}
\usepackage{longtable}
\hypersetup{
    colorlinks,
    citecolor=blue,
    filecolor=black,
    linkcolor=red,
    urlcolor=blue
}
\usepackage[round]{natbib}

%% Comments

\usepackage{color}

\newif\ifcomments\commentstrue

\ifcomments
\newcommand{\authornote}[3]{\textcolor{#1}{[#3 ---#2]}}
\newcommand{\todo}[1]{\textcolor{red}{[TODO: #1]}}
\else
\newcommand{\authornote}[3]{}
\newcommand{\todo}[1]{}
\fi

\newcommand{\wss}[1]{\authornote{blue}{SS}{#1}} 
\newcommand{\plt}[1]{\authornote{magenta}{TPLT}{#1}} %For explanation of the template
\newcommand{\an}[1]{\authornote{cyan}{Author}{#1}}

%% Common Parts

\newcommand{\progname}{ProgName} % PUT YOUR PROGRAM NAME HERE %Every program
                                % should have a name


\begin{document}

\title{System Verification and Validation Plan for Truss} 
\author{Ting-Yu Wu}
\date{\today}
	
\maketitle

\pagenumbering{roman}

\section{Revision History}

\begin{tabularx}{\textwidth}{p{3cm}p{2cm}X}
\toprule {\bf Date} & {\bf Version} & {\bf Notes}\\
October 29,~2020 & 1.0 & Initial version of VnV plan\\
\bottomrule
\end{tabularx}

\newpage

\tableofcontents

\listoftables

\listoffigures

\newpage

\section{Symbols, Abbreviations and Acronyms}

\renewcommand{\arraystretch}{1.2}
\begin{tabular}{l l} 
  \toprule		
  \textbf{symbol} & \textbf{description}\\
  \midrule 
  FR & Functional Requirements\\
  NFR & Nonfunctional Requirements\\
  R & Requirements\\
  SRS & Software Requirements Specification\\
  VnV & Verification and Validation\\
  \bottomrule
\end{tabular}\\


\newpage

\pagenumbering{arabic}

This document provides an overview of the Verification and Validation (VnV) 
plan for Truss. The general information is introduced in section \ref{Geninf}. 
Verification plans and test description are in section \ref{verplan} and 
section \ref{sysdescript}, respectively.

\section{General Information} \label{Geninf}

\subsection{Summary}

The software being test in this document is Truss. Users can input the external 
force and the structure of the truss, the software will calculate all the 
internal forces within truss members and ouput the result with a .txt file. 


\subsection{Objectives}
The objective of the VnV plan is to verify the FR and NFR described in the SRS. 
We will test all the functional requirements and nonfunctional requirements in 
Section \ref{sysdescript}. The most important goals are building confidence 
in the software correctness and increasing the reliability of the software. 

\subsection{Relevant Documentation}

\begin{itemize}
	\item 
	\href{https://github.com/tingyuw/cas741/blob/master/docs/SRS/SRS.pdf}{SRS} 
	for Truss
\end{itemize}
\an{Other documents will be included after building}


\section{Plan} \label{verplan}
This section lists the VnV plan of Truss. Section \ref{VnVteam} introduces the 
members of the VnV team. Verification plans of SRS, design, and implementation 
are covered in section \ref{SRSplan}, section \ref{designplan}, and section 
\ref{implplan}, respectively. Section \ref{autotool} outlines the tools that 
are used for automated testing. Section \ref{sftwareplan} outlines the 
validation plan of the software.

\subsection{Verification and Validation Team} \label{VnVteam}
This section lists the members of verification and validation team.
\begin{itemize}
	\item Ting-Yu Wu
	\item Dr. Spencer Smith
	\item Dr. Jacques Carette
	\item Tiago de Moraes Machado reviews the whole project
	\item Xuanming Yan reviews the SRS
	\item Mohamed AbuElAla reviews the VnV plan
\end{itemize}

\subsection{SRS Verification Plan} \label{SRSplan}
The SRS can be reviewed by the author's supervisors and classmates from the 
class. Reviewers can give feedbacks and revision suggestions to the author by 
creating issues on GitHub. It is author's responsibility to check the submitted 
issues regularly and make necessary revisions.

\subsection{Design Verification Plan} \label{designplan}
The design of this software will be verified by testing functional requirements 
and nonfunctional requirements. Test cases for functional requirements are 
listed in section \ref{testfr}, and test cases for nonfunctional requirements 
are listed in section \ref{testnfr}.

\subsection{Implementation Verification Plan} \label{implplan}
The implementation verification plan includes the followings:
\begin{itemize}
	\item Code walkthroughs. The rubber duck testing method will be 
	implemented. The procedure involve explaining the code line by line to the 
	duck and go into detail of what the code is supposed to do. 
	\item Expert review. This verification will be performed by individuals in 
	the verification and validation team, as listed in section \ref{VnVteam}, 
	by paying close attention and looking for potential implementation errors. 
	\item Unit testing. Tool we used for unit testing is PyUnit, the python 
	unit testing framework. More details are outlined in section 
	\ref{unitdescript}.
\end{itemize}

\subsection{Automated Testing and Verification Tools} \label{autotool}
We implement PyUnit for automated unit testing. 

\an{More details and tools will be included later}\\
\wss{What tools are you using for automated testing.  Likely a unit testing
  framework and maybe a profiling tool, like ValGrind.  Other possible tools
  include a static analyzer, make, continuous integration tools, test coverage
  tools, etc.  Explain your plans for summarizing code coverage metrics.
  Linters are another important class of tools.  For the programming language
  you select, you should look at the available linters.  There may also be tools
  that verify that coding standards have been respected, like flake9 for
  Python.}
\wss{The details of this section will likely evolve as you get closer to the
  implementation.}

\subsection{Software Validation Plan} \label{sftwareplan}
The software will be validated by testing the correctness of outputs, which is 
covered in section \ref{outverify}.

\section{System Test Description} \label{sysdescript}
	
\subsection{Tests for Functional Requirements} \label{testfr}
The functional requirements are described in the SRS. Truss shall verify that 
the inputs are valid and the calculated outputs are correct.
FR1 and FR2 will be tested in section \ref{inverify}. FR3 and FR4 will be 
tested in section \ref{outverify}.

\subsubsection{Input Verification} \label{inverify}
According to FR1 and FR2 in the SRS, Truss shall take inputs from users and 
verify whether the inputs meet the data constraints, as described in the 
section 4.2.6 in SRS. If the input values are incorrect or out of bounds, the 
software shall display an error message.
		
\paragraph{Input Verification test}

\begin{enumerate}

\item{Valid inputs\\}

Control: Automatic

Initial State: Truss is started and running

Input: $F_1 = 500$, $x_1 = 3$, $x_2 = 7$, $\theta_1 = 30$, $\theta_2 = 30$

Output: Generate an output file and display a success message

Test Case Derivation: Successfully generate an output file

How test will be performed: Automated system test

\item{Invalid distance\\}

Control: Automatic
					
Initial State: Truss is started and running
					
Input: $x_1 = 0$ ; $x_1 = -3$ ; $x_2 = 0$ ; $x_2 = -3$
					
Output: An error message of "Please input positive value for the distance"

Test Case Derivation: Successfully display the error message

How test will be performed: Automated system test

\item{Invalid angle\\}

Control: Automatic

Initial State: Truss is started and running

Input: $\theta_1 = 0$ ; $\theta_1 = 90$ ; $\theta_1 = -45$ ; $\theta_2 = 0$ ; 
$\theta_2 = 90$ ; $\theta_2 = -45$

Output: An error message of "Please input the value greater than 0 and less 
than 90 for the angle"

Test Case Derivation: Successfully display the error message

How test will be performed: Automated system test

\end{enumerate}
\an{More specific input cases will be included later.}

\subsubsection{Output Verification} \label{outverify}
According to FR3 and FR4 in the SRS, Truss shall calculate equations and output 
the correct values for all internal forces. 

\paragraph{Output Correctness test}

\begin{enumerate}
	
	\item{Simple case\\}
	
	Control: Automatic
	
	Initial State: Not applicable
	
	Input: $F_1 = 500$, $x_1 = 3$, $x_2 = 7$, $\theta_1 = 30$, $\theta_2 = 30$
	
	Output: $F_{AC} = -700$, $F_{AD} = 606.22$, $F_{BC} = -300$, $F_{BD} = 
	259.81$, $F_{CD} = 500$
	
	Test Case Derivation: Successfully calculate the expected outputs
	
	How test will be performed: Automated system test
	
	\item{Correctness of output file\\}
	
	Control: Automatic
	
	Initial State: Not applicable
	
	Input: the output file
	
	Output: the relative error between each value at each time step
	
	Test Case Derivation: Successfully calculate the expected outputs
	
	How test will be performed: Automated system test
\end{enumerate}

\subsection{Tests for Nonfunctional Requirements} \label{testnfr}
The nonfunctional requirements are described in the SRS. All the qualities of 
Truss will be tested in the following section. Some requirements can be 
measured by the grade sheet, such as table \ref{Undgradesheet} for 
understandability. In some cases a superscript * is used to indicate 
that a response of this type should be accompanied by explanatory 
text~\cite{Smithetal2018}.\\
NFR1 correctness and NFR2 verifiability will be tested in section \ref{candv}. 
NFR3 understandability, NFR4 portability, NFR5 maintainability, and NFR6 
reliability will be tested in section \ref{under}, section \ref{port}, section 
\ref{main}, and section \ref{reliab}, respectively.


\subsubsection{Correctness and Verifiability} \label{candv}
The correctness test covers the NFR1, and the verifiability test covers the
NFR2. Both tests are to ensure that the software meets the SRS, and they can 
be assess through this document.
		
\subsubsection{Understandablility} \label{under}
The understandability test covers the NFR3.
\paragraph{Understandablility test}

\begin{enumerate}

\item{Code review\\}

Type: Manual
					
Initial State: Not applicable
					
Input/Condition: Review the source code
					
Output/Result: How easy can a new developer understand the source code
					
How test will be performed: Understandability can be measured by the grade 
sheet in Table \ref{Undgradesheet}

\end{enumerate}

\begin{longtable}{l l}
	\begin{tabular}{l l} 
	\toprule		
	\textbf{Questions} & \textbf{Answer set}\\
	\midrule 
	Consistent indentation and formatting style? & \{yes, no, n/a\}\\
	Explicit identification of a coding standard? & \{yes*, no, n/a\}\\
	Are the code identifiers consistent, distinctive, and meaningful? & \{yes, 
	no*, n/a\} \\
	Are constants (other than 0 and 1) hard-coded into the program? & \{yes, 
	no*, n/a\} \\
	Comments are clear, indicate what is being done, not how? & \{yes, no*, 
	n/a\} \\
	Is the name/URL of any algorithms used mentioned? & \{yes, no*, n/a\} \\
	Parameters are in the same order for all functions? & \{yes, no*, n/a\} \\
	Is code modularized? & \{yes, no*, n/a\} \\
	Descriptive names of source code files? & \{yes, no*, n/a\} \\
	Is a design document provided? & \{yes*, no, n/a\} \\
	Overall impression? & \{1 .. 10\} \\
	\bottomrule
	\caption{Understandability grade sheet} \label{Undgradesheet} \\
\end{tabular}\\
\end{longtable}

\subsubsection{Portablility} \label{port}
The portability test covers the NFR4.
\paragraph{Portablility test}

\begin{enumerate}
	
	\item{Portability on Windows system\\}
	
	Type: Manual
	
	Initial State: Truss has been successfully installed on a Windows system
	
	Input/Condition: Perform basic functions of the software
	
	Output/Result: Successfully perform the functions
	
	How test will be performed: Portability can be measured by the grade 
	sheet in Table \ref{Porgradesheet}. It will be performed by test team 
	manually
	
	\item{Portability on Linux system\\}
	
	Type: Manual
	
	Initial State: Truss has been successfully installed on a Linux system
	
	Input/Condition: Perform basic functions of the software
	
	Output/Result: Successfully perform the functions
	
	How test will be performed: Portability can be measured by the grade 
	sheet in Table \ref{Porgradesheet}. It will be performed by test team 
	manually
	
		
	\item{Portability on MacOS system\\}
	
	Type: Manual
	
	Initial State: Truss has been successfully installed on a MacOS system
	
	Input/Condition: Perform basic functions of the software
	
	Output/Result: Successfully perform the functions
	
	How test will be performed: Portability can be measured by the grade 
	sheet in Table \ref{Porgradesheet}. It will be performed by test team 
	manually
	
	\begin{longtable}{l l}
		\begin{tabular}{l l} 
			\toprule		
			\textbf{Questions} & \textbf{Answer set}\\
			\midrule 
			What platforms is the software advertised to work on? & \{Type of 
			OS\}\\
			Are special steps taken in the source code to handle portability? & 
			\{yes*, no, n/a\}\\
			Is portability explicitly identified as NOT being important? & 
			\{yes, no, n/a\} \\
			Convincing evidence that portability has been achieved? & 
			\{yes*, no,\} \\		
			Overall impression? & \{1 .. 10\} \\
			\bottomrule
			\caption{Portability grade sheet} \label{Porgradesheet} \\
		\end{tabular}\\
	\end{longtable}

\end{enumerate}

\subsubsection{Maintainability} \label{main}
The maintainability test covers the NFR5.
\paragraph{Maintainability test}

\begin{enumerate}
	
	\item{Version control\\}
	
	Type: Manual
	
	Initial State: Not applicable
	
	Input/Condition: Existing Truss system
	
	Output/Result: Exist multiple versions of the system
	
	How test will be performed: Test team will perform manually to 
	check whether there exists a history of multiple versions of the software
	
	\item{Issue tracking\\}
	
	Type: Manual
	
	Initial State: Not applicable
	
	Input/Condition: Existing Truss system
	
	Output/Result: Implementing issue tracking tools
	
	How test will be performed:  Test team will perform manually to 
	check whether a issue tracking tool is implemented
	
\end{enumerate}

\subsubsection{Reliability} \label{reliab}
The reliability test covers the NFR6.
\paragraph{Reliability test}

\begin{enumerate}
	
	\item{Software running\\}
	
	Type: Manual
	
	Initial State: Not applicable
	
	Input/Condition: Existing Truss system
	
	Output/Result: Successfully operate the software
	
	How test will be performed: Test team will run the software manually to 
	check whether it breaks during installation and operation
	
	\item{Performance time\\}
	
	Type: Manual
	
	Initial State: Not applicable
	
	Input/Condition: Truss is started and running
	
	Output/Result: Time duration of the software to perform required functions
	
	How test will be performed: The test will be performed by test team manually
	
\end{enumerate}

\subsection{Traceability Between Test Cases and Requirements}

\begin{table}[h!]
	\centering
	\begin{tabular}{|c|c|c|c|c|c|c|c|c|c|c|}
		\hline
		& R1 & R2 & R3 & R4 & R5 & R6 & R7 & R8 & R9 & R10\\
		\hline
		\ref{inverify}    &X &X & & & &X & & & &\\ \hline
		\ref{outverify}   & & &X &X &X &X & & & &\\ \hline
		\ref{candv}       & & & & &X &X & & & &\\ \hline
		\ref{under}       & & & & & & &X & & &\\ \hline
		\ref{port}        & & & & & & & &X & &\\ \hline
		\ref{main}        & & & & & & & & &X &\\ \hline
		\ref{reliab}      & & & & & & & & & &X\\ \hline
	\end{tabular}
	\caption{Traceability Between Test Cases and Requirements}
	\label{Table:A_trace}
\end{table}





\section{Unit Test Description} \label{unitdescript}

\wss{Reference your MIS and explain your overall philosophy for test case
  selection.}  
\wss{This section should not be filled in until after the MIS has
  been completed.}

\subsection{Unit Testing Scope}

\wss{What modules are outside of the scope.  If there are modules that are
  developed by someone else, then you would say here if you aren't planning on
  verifying them.  There may also be modules that are part of your software, but
  have a lower priority for verification than others.  If this is the case,
  explain your rationale for the ranking of module importance.}

\subsection{Tests for Functional Requirements}

\wss{Most of the verification will be through automated unit testing.  If
  appropriate specific modules can be verified by a non-testing based
  technique.  That can also be documented in this section.}

\subsubsection{Module 1}

\wss{Include a blurb here to explain why the subsections below cover the module.
  References to the MIS would be good.  You will want tests from a black box
  perspective and from a white box perspective.  Explain to the reader how the
  tests were selected.}

\begin{enumerate}

\item{test-id1\\}

Type: \wss{Functional, Dynamic, Manual, Automatic, Static etc. Most will
  be automatic}
					
Initial State: 
					
Input: 
					
Output: \wss{The expected result for the given inputs}

Test Case Derivation: \wss{Justify the expected value given in the Output field}

How test will be performed: 
					
\item{test-id2\\}

Type: \wss{Functional, Dynamic, Manual, Automatic, Static etc. Most will
  be automatic}
					
Initial State: 
					
Input: 
					
Output: \wss{The expected result for the given inputs}

Test Case Derivation: \wss{Justify the expected value given in the Output field}

How test will be performed: 

\item{...\\}
    
\end{enumerate}

\subsubsection{Module 2}

...

\subsection{Tests for Nonfunctional Requirements}

\wss{If there is a module that needs to be independently assessed for
  performance, those test cases can go here.  In some projects, planning for
  nonfunctional tests of units will not be that relevant.}

\wss{These tests may involve collecting performance data from previously
  mentioned functional tests.}

\subsubsection{Module ?}
		
\begin{enumerate}

\item{test-id1\\}

Type: \wss{Functional, Dynamic, Manual, Automatic, Static etc. Most will
  be automatic}
					
Initial State: 
					
Input/Condition: 
					
Output/Result: 
					
How test will be performed: 
					
\item{test-id2\\}

Type: Functional, Dynamic, Manual, Static etc.
					
Initial State: 
					
Input: 
					
Output: 
					
How test will be performed: 

\end{enumerate}

\subsubsection{Module ?}

...

\subsection{Traceability Between Test Cases and Modules}

\wss{Provide evidence that all of the modules have been considered.}
				
\newpage

\bibliographystyle {plainnat}
\bibliography {ref}

%\newpage


%\section{Appendix}

%This is where you can place additional information.

%\subsection{Symbolic Parameters}

%The definition of the test cases will call for SYMBOLIC\_CONSTANTS.
%Their values are defined in this section for easy maintenance.

%\subsection{Usability Survey Questions?}

%\wss{This is a section that would be appropriate for some projects.}


\end{document}
