\documentclass{article}

\usepackage{tabularx}
\usepackage{booktabs}

\title{CAS 741: Problem Statement\\Truss}

\author{Ting-Yu Wu}

\date{\today}

%% Comments

\usepackage{color}

\newif\ifcomments\commentstrue

\ifcomments
\newcommand{\authornote}[3]{\textcolor{#1}{[#3 ---#2]}}
\newcommand{\todo}[1]{\textcolor{red}{[TODO: #1]}}
\else
\newcommand{\authornote}[3]{}
\newcommand{\todo}[1]{}
\fi

\newcommand{\wss}[1]{\authornote{blue}{SS}{#1}} 
\newcommand{\plt}[1]{\authornote{magenta}{TPLT}{#1}} %For explanation of the template
\newcommand{\an}[1]{\authornote{cyan}{Author}{#1}}


\begin{document}

\maketitle

\begin{table}[hp]
\caption{Revision History} \label{TblRevisionHistory}
\begin{tabularx}{\textwidth}{llX}
\toprule
\textbf{Date} & \textbf{Developer(s)} & \textbf{Change}\\
\midrule
September 21, 2020 & Ting-Yu Wu & Initial Draft\\
September 27, 2020 & Ting-Yu Wu & Updates according to issue \#1\\
\bottomrule
\end{tabularx}
\end{table}

\section*{Problem}
A truss is a framework that could hold something up, supporting bridges, 
roofs, or other structures. This project will focus on solving truss bridge 
problems. Before operating a bridge, there are lots of testing and evaluations 
need to be done, one of them is an analysis of trusses. We are trying to figure 
out whether a bridge is safe enough to support various loads it encounters 
(e.g., the weight of vehicles crossing it) by analyzing the stresses and 
unknown forces acting in truss members. Knowing both motions and forces within 
the trusses prepares us for a better understanding of how stable the 
architecture is. The inputs of the software are the structures of the 
bridge, external reacting forces, and load forces. The outputs are all the 
internal forces of truss members and their stress distribution (tension and 
compression).\\

We will implement the Drasil project on this software. Following is the 
homepage of Drasil: https://github.com/JacquesCarette/Drasil/. 

\section*{Context}
Specific stakeholders may include students, professors, researchers in the 
field of mechanical and civil engineering, individuals interested in solving 
forces of truss members, and all contributors to the Drasil project. The 
software will be compatible with a variety of operating system environments, 
including Windows, Linux and MacOS. 


\end{document}
